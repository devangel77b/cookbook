\chapter{Test}
\label{ch:Test}
\index{test}
\marginnote{This chapter was written by \smallcaps{Ben Umayam}, who has a lot to say about his culinary self in a blurb right here. When not writing cookbooks, he enjoys stuff.}
\marginnote{When using this chapter stand-alone, please cite the following: \bibentry{ch:test}}
% place on first page near bottom?
% chapter abstract? or abstract section? 

\section{Wonderful things you can do with sugar}
\newthought{Desserts are tasty.}  Wondrous confections, fattened with cream, and flavored with the delicate, the exotic, or the bold.  Yum.

Back in the highlands of Ilocos, my mother used to prepare Yorkshire Pudding at the end of the rainy season, when the carabao came down from the highlands to ride in jeepneys and go to Shoemart.  To this day, the floury smell always reminds me of those days we would have all the cousins stand on a piece of paper to trace their feet so we'd know what size chinellas to get. 

Cookbooks like to have a kitschy chapter of personal anecdotes connected to the food.  We can use these sorts of stories also to force the pagination to be correct, so that recipes appear on their own page in their entirety.

These days, cooks often take shortcuts in preparing their personal anecdotes but with some simple planning it doesn't have to be so!  Mike and I like to collect our kitschy anecdotes on index cards as they come to us throughout the year; then when we need one we simply browse through the card file and pull out one that fits.  Tearing up the card when we use it ensures we don't use one more than once. 
\vfill

\clearpage
% RECIPE HERE
% Serving suggestions, etc
\marginnote{Yorkshire pudding is tasty.  Try serving with sushi or with burritos, three great tastes that taste great together.}
% index terms here
\index{pudding}\index{pudding!Yorkshire}\index{Yorkshire pudding}
% the actual recipe
\begin{recipe}{Yorkshire pudding}{serves 4}{\SI{1}{\hour}}
\ingredient{\SI{0.5}{pint}}{milk}
\ingredient{\SI{2}{oz}}{butter}
\ingredient{\SI{5}{oz}}{self-rising flour}
Heat the milk and butter until nearly boiling.  Add flour and allow to seeth over.
\ingredient{\num{3}}{eggs}
\ingredient{to taste}{salt and pepper}
Add the remaining eggs and whisk again.  Cook at \SI{200}{C} for \SI{1}{\hour}.
\end{recipe}
% photo
\begin{figure*}[h!]
\centering
\includegraphics[width=\textwidth]{figures/yorkshire-pudding.jpg}
\caption{The mark of a good Ilocano Yorkshire pudding is that it is puffy.}
%\label{fig:yorkshire-pudding}
\end{figure*}
%% END RECIPE


% How to do a citation:
%Solids are evil. That is why the Dominion \index{Dominion War} wages war against them \cite{Son-of-Mohg:2388}.

% How to do chapter acknowledgements
%\section*{Acknowledgements}
%Chapter acknowledgements here.


